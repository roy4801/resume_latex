% !TEX program = xelatex

\documentclass{resume}
\usepackage{xeCJK}
\setCJKmainfont{Noto Sans CJK TC}

\begin{document}
\pagenumbering{gobble} % suppress displaying page number

\name{鍾秉桓 Roy Zhong}

\basicInfo{
  \email{a82611141@gmail.com} \textperiodcentered\ 
  \phone{(+886) 0901379848} \textperiodcentered\ 
  \linkedin[roy4801]{https://www.linkedin.com/in/roy4801}}

%%%%%%%%%%%%%%%%%%%%%%% 教育
\section{\faGraduationCap\ 學歷}

\datedsubsection{\textbf{}輔仁大學資訊工程學系 \textit{學士}}{2017 -- 2021}

%%%%%%%%%%%%%%%%%%%%%%% 經驗
% \section{\faUsers\ Experience}

% \datedsubsection{\textbf{FLAG Inc.} California, America}{2012 -- Present}
% \role{Summer Intern}{Manager: xxx}
% Brief introduction: xxx.
% \begin{itemize}
%   \item Implemented xxx feature
%   \item Optimized xxx 5\%
%   \item xxx
% \end{itemize}

%%%%%%%%%%%%%%%%%%%%%%% 專案
\section{\faTasks\ 專案}

\datedsubsection{\textbf{Rish Engine}}{Apr. 2020 -- Present}
\role{Group Leader}{Group Projects}
{\small https://github.com/rishteam/dod} \\
一個簡單的 2D 遊戲引擎, 使用 C++, SFML
\begin{itemize}
  \item Entity Component System design
  \item Simple editor allowing developer to make their own scene and able to save in own format
  \item Support scripting language
\end{itemize}

\datedsubsection{\textbf{選課小幫手 CoursePicker}}{Nov. 2019 -- Dec. 2019}
\role{Group Leader}{Group Projects}
{\small https://github.com/rishteam/db-back} \\
一個給輔大學生模擬選課的小工具
\begin{itemize}
  \item 瀏覽課程, 篩選, 新增/刪除課程, 排課
  \item RESTful API, Android App
\end{itemize}

\datedsubsection{\textbf{RishOJ - Online Judge}}{Apr. 2019 -- Jun. 2019}
\role{Group Leader}{Group Projects}
一個給程式競賽的線上程式碼繳交系統, {\small https://github.com/FjuOnlineJudge/oj}
\begin{itemize}
  \item 使用 Python, flask, MySQL
  \item 列表, 篩選競賽題目, 提交程式碼, 查看結果, 排名, ... 等.
\end{itemize}

\datedsubsection{\textbf{Rifleman}}{Feb. 2019 -- Apr. 2019}
\role{Group Leader}{Group Projects}
一個 2D 的射擊小遊戲, {\small https://github.com/william31212/Pygame}
\begin{itemize}
  \item RHFramework 一個簡單2D遊戲框架 {\small https://github.com/roy4801/RHframework}
  \item 使用 Python, Pygame, OpenGL, 使用 Tiled 編輯地圖 
\end{itemize}

%%%%%%%%%%%%%%%%%%%%%%% 技能
\section{\faCogs\ 技能}
\begin{itemize}[parsep=0.5ex]
  \item Programming Languages: C++ > Python > C 
  \item Scripting Languages: shell script
\end{itemize}

%%%%%%%%%%%%%%%%%%%%%%% 獲獎
\section{\faTrophy\ 獲獎}

\subsection{程式競賽}
\datedline{\textit{Gold Place}, ICPC Taiwan NCPU}{Jun. 2020}
\datedline{\textit{Bronze Place}, ICPC Asia Taipei-Hsinchu Regional}{Nov. 2019}
\datedline{\textit{Silver Place}, ICPC Taiwan NCPU}{Jun. 2019}
\datedline{\textit{\nth{2} Place}, USCOJ}{2019}

\subsection{資安競賽}
\datedline{\textit{Qualified}, Advanced Information Security Summer School}{2020, 2019, 2018}
\datedline{\textit{\nth{9} /36}, Taiwan Cyber Security Competition}{2020}

%%%%%%%%%%%%%%%%%%%%%%% 社群
\section{\faUsers\ 社群}

\subsection{社團}
\datedline{社長,  FJCU NISRA {\small(資訊安全)\ https://www.nisra.net/}}{Jul. 2019 -- Jul. 2020}
\datedline{講師, 成員,  FJCU NISRA }{Jul. 2017 -- Present}
\datedline{講師, 成員,  FJCU CPC {\small(程式競賽)}}{Jul. 2019 -- Present}
SlideShare: https://www.slideshare.net/ssuserbafcfa

\subsection{開源社群}
\datedline{場務組, HITCON 台灣駭客年會}{2018}

%%%%%%%%%%%%%%%%%%%%%%% 其他
\section{\faInfo\ 其他}
\begin{itemize}[parsep=0.5ex]
  \item Blog: http://blog.roy4801.tw
  \item GitHub: https://github.com/roy4801
  \item 語言:
    \subitem 中文 - Native speaker
    \subitem English - TOEIC 895
\end{itemize}

%% Reference
% \newpage
% \bibliographystyle{IEEETran}
% \bibliography{mycite}
\end{document}
